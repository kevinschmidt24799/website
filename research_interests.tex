\documentclass{article}
\usepackage{amsmath}
\usepackage{amssymb}

\begin{document}

\section*{Research Interests}

I am broadly interested in group-theoretic and linear algebraic approaches to combinatorics, including spectral graph theory, representation theory, and discrete isoperimetric inequalities. I particularly enjoy problems that admit multiple interpretations and can be approached from different perspectives.

\section*{Specific Research Directions}

\subsection*{Spectral Graph Theory}

I'm interested in eigenvalue bounds for graphs, including the supremum of
$$\{\lambda_i(G)/|G|\}$$
over all graphs $G$, sums and products of graph eigenvalues, and the distribution of eigenvalues in random graphs. Related to this, I'm curious about the proportion of graphs that are cospectral---while almost all trees are cospectral, trees constitute a negligible fraction of all graphs, and the abundance of edges in typical graphs makes it harder to swap cospectral components.

\subsection*{Arithmetic Progressions with Digit Sum Constraints}

What is the longest arithmetic progression among numbers with a fixed digit sum in a given base? For instance, in base 10 with digit sum 14, the longest appears to be $59 + 99k$ for $k = 0, \ldots, 59$ (length 60), while in base 2 with digit sum 3, it's $7 + 7k$ for $k = 0, \ldots, 7$ (length 8). A general pattern emerges with additional casework, but proving optimality remains challenging.

\subsection*{Strongly Regular Graphs}

I'm interested in existence questions for strongly regular graphs with specific parameters, including Conway's 99-graph problem: does there exist a strongly regular graph with parameters $(99, 14, 1, 2)$?

\subsection*{Multiple Interpretations of Mathematical Objects}

Matrices can be viewed as graphs, linear operators, point configurations, two-person zero-sum games, quadratic forms, and more. I'm fascinated by which facts become transparent under different interpretations, how choosing the right perspective can solve otherwise difficult problems, and how natural operations in one interpretation become unnatural in another. For example, how do a matrix game and its square relate as games?

I'm also interested in interpreting probabilities of $n$ events as partitions of an $n$-dimensional unit cube. When events are independent, the cube slices fit together proportionately. I'd like to find a geometric interpretation of the Lov\'asz Local Lemma using this framework.

\subsection*{Algebraic Graph Theory and Representation Theory}

The connection between eigenvalues of Cayley graph adjacency matrices and irreducible representations of groups, particularly for non-abelian groups, is an area where I'm still building foundational knowledge and exploring potential research directions.

\subsection*{Geometric Covering Problems}

What is the largest $k$ such that any $k$ points in the plane can be covered with disjoint unit disks? Currently
$$13 \leq k \leq 45,$$
with the lower bound from probabilistic methods and the upper bound from lattice constructions. I'm particularly interested in improving the lower bound.

\subsection*{Applied Combinatorics and Algorithms}

I enjoy thinking about mathematical structures in games and puzzles. For instance, in Codenames, can a perfect team devise a pre-emptive strategy to solve the board in one round? (I believe not, but two rounds seems possible.) Could LLM word embeddings provide effective semantic hints, and would such hints be human-interpretable given the high-dimensional, sparse nature of the space?

\subsection*{Probabilistic Methods and Fair Division}

How can we break a three-way tie with a fair coin? No deterministic finite procedure exists, but which method minimizes expected flips? I've solved this specific case, but the general problem of simulating one probability distribution from another is harder. Even when deterministic methods exist, is the fastest method always deterministic? (I suspect not.) This connects to encoding digits of fractions in various bases. This is related to the Knuth-Yao optimization. 

\end{document}